\chapter{JSOUP API}
\vspace{1cm}
Jsoup este o librărie Java pentru lucrul cu elemente specifice HTML, cuprinzând metode speciale pentru a manipula datele utilizând cele mai populare metode DOM, CSS si jquery.\newline
Jsoup :
\begin{itemize}
	\item găsește și extrage date utilizâd DOM\footnote{Document Object Model (DOM) este o convenție cross-platform și independentă de limbajul utilizat, folosită pentru reprezentarea și interacționarea cu obiecte în documente HTML, XHTML, și XML.\cite{16}} traversal sau selectori CSS\footnote{Cascading Style Sheets (CSS) este un limbaj de stil utilizat pentru design-ul și formatarea unui document scris într-un limbaj markup.}
	\item parsează pagini HTML prin intermediul URL-ului, a unui fișier HTML sau a unui string.
	\item manipulează elementele HTML, atributele și textul.
	\item curăță conținutul prezentat de utilizatori utilizănd o listă albă pentru a preveni XSS\footnote{Cross-site scripting (XSS) este un tip de vulnerabilitate a securității unui computer, în general găsită în aplicațiile Web. }
	\item datele de ieșire reprezintă HTML ordonat
	\item jsoup este proiectat să se ocupe de toate tipurile de HTML care există; de la curat și valid, la taguri-soup invalide; jsoup va crea un arbore de parsare sensibil.
\end{itemize}\newpage
\begin{exmp} Exemplu de utilizare a API-ului jsoup\newline
Descarcă pagina Wikipedia, o parseaza ca un DOM și selectează titlurile din sectiunea de stiri intr-o lista de Elements:\newline
Document doc = Jsoup.connect("http://en.wikipedia.org/").get();\newline
Elements newsHeadlines = doc.select("\#mp-itn b a");\newline
\end{exmp}
\textbf{Open source}\newline
Jsoup este un proiect open source distribuit sub licența MIT\footnote{Licența MIT este o licență gratuită având originea la Insitutul de Tehnologie din Massachusetts \cite{17} }. Codul sursă este disponibil pe GitHub.\newline

Pentru a utiliza jsoup, trebuie descărcată arhiva jar utilizând url-ul https://jsoup.org/download, versiunea 1.8.2.\newline
Aceasta trebuie dezarhivată și importată în proiectul în care umrează să fie folosit.\newline

Dacă se utilizează \textbf{Maven \footnote{Maven este un instrument automat utilizat în general în proiectele Java. Maven descarcă dinamic librării Java și plug-in-uri Maven din unul sau mai multe depozite precum Maven 2 Central Repository, și le stochează în cache-ul local.}} pentru a manageria dependențele dintr-un proiect Java(lucru recomandat), nu este nevoie să descărcăm arhiva jar, ci este suficient să punem următoarea bucată de cod în fișierul POM, secțiunea <dependencies>: \newline

<dependency>\newline
  <!-- jsoup HTML parser library @ http://jsoup.org/ -->\newline
  <groupId>org.jsoup</groupId>\newline
  <artifactId>jsoup</artifactId>\newline
  <version>1.8.2</version>\newline
</dependency>\newline

\textbf{Dependințe}\newline
Jsoup nu are dependințe.\newline

Jsoup rulează cu Java începând de la versiunea 1.5, cu Scala, Android, OSGi și Google App Engine.\newline

Pentru a învăța cum se utilizează librăria jsoup este recomandat să parcurgem Cookbook-ul care se găsește pe https://jsoup.org/cookbook/ care cuprinde:\cite{18}\newline
\textbf{Introducere}
\begin{enumerate}
	\item Parsarea și parcurgerea unui Document
\end{enumerate}
\newpage
\textbf{Input}
\begin{enumerate}
	\item Parsarea unui document preluat dintr-un String
	\item Încărcarea unui Document printr-un URL
	\item Încărcarea unui Document dintr-un fișier
\end{enumerate}

\textbf{Extragerea de date}
\begin{enumerate}
	\item Utilizarea de metode DOM pentru a naviga într-un document
	\item Utilizarea de selectori sintactici pentru a căuta elemente
	\item Extragerea de atribute, text și HTML din elemente		
	\item Lucrul cu URL-uri
	\item Un exemplu de program - printarea unor link-uri
\end{enumerate}

\textbf{Modificarea datelor}
\begin{enumerate}
	\item Setarea valorilor unor atribute
	\item Setarea HTML pentru un element
	\item	Setarea conținutului text al elementelor
\end{enumerate}

\textbf{Curățarea HTML}
\begin{enumerate}
	\item Eliminarea surselor HTML nesigure pentru prevenirea XSS
\end{enumerate}

\textbf{Try jsoup}\newline
Try jsoup este un demo online, interactiv care permite vizualizarea parsării unui HTML într-un DOM și testarea interogărilor CSS.
