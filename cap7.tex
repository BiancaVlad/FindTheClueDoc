\chapter{Capitolul 8}
\section{Ce este Azure?}
\subsection{Prezentare generală a cloud computing}
Un data-centru local necesită gestionarea multor detalii, cum ar fi cumpărare și instalarea de hardware, gestionarea virtualizării, instalarea sistemului de operare și  a oricăror aplicații necesare, configurarea rețelei (inclusiv a firelor necesare pentru funcționare), configurarea firewall-ului dar și configurarea stocării datelor. După ce toate acestea au fost făcute, intervine apoi responsabilitatea pentru menținerea sistemului pentru întreg ciclul de viață. Aceste lucruri înseamnă costuri foarte mari pentru hardware dar și costuri pentru întreținere.\newline
Cloud computing vine cu o alternative modernă la un data-centru local traditional. Un furnizor public de servicii cloud este complet responsabil pentru achiziționarea și întreținerea de hardware, și de obicei oferă o gamă largă de servicii de platformă ce pot fi folosite. Se pot alege serviciile și resursele hardware și software ce sunt necesare, iar partea cea mai importantă este că pentru aceste servicii și resurse se va plăti doar atunci când vor fi folosite.\newline
Mediile cloud oferă de obicei un portal online, ușurând astfel viața utilizatorilor să administreze calculi, spații de stocare, rețele și resurse de aplicații. De exemplu, un utilizator poate folosi portalul pentru a crea o configurație de mașină virtuală (VM) specificând următoarele: dimensiunea nodului de calcul (cu referire la CPU, RAM și disk local), sistemul de operare, software preinstalat, configurația rețelei, sau locația nodului. Utilizatorul poate apoi lansa mașina virtuală bazată pe aceste configurații și în câteva minute poate accesa nodul de calcul lansat. \newline
Pe lângă serviciul public de cloud descris anterior, mai există cloud-uri private și hibride. Într-un cloud privat, se creaza un mediu de cloud în propriul data-centru și se poate oferi access de tip „serviciu propriu” pentru a calcula resursele utilizatorilor dintr-o organizație. Aceast mecanism oferă o simulare a unui cloud public utilizatorilor, dar administratorul cloud-ului va fi complet responsabil de achiziționarea și întreținerea serviciilor hardware și software oferite. Un cloud hibrid integrează cloud-uri publice și private, oferind  posibilitatea de a găzdui volume de muncă în cele mai potrivite locații. De exemplu, se poate găzdui un website foarte mare în cloud-ul public și se poate asocia la o bază de dată securizată găzduită în cloud-ul privat.\newline
Microsoft oferă suport pentru cloud-uri publice, private și hibride. Microsot Azure este un cloud public.\newline

\subsection{Comparație între local și Azure}
Cu o infrastructură locală, administratorul deține controlul total asupra hardware-ului și a software-ului ce se instalează. Istoric, acești factori au dus la decizii de cumpărare a hardware-ului orientate către scalare în sus (scaling-up); și anume achiziționarea de servere cu mai multe nuclee pentru a satisface nevoile de performanță. Cu Azure, se pot instala aplicații doar pe hardware-ul oferit de Microsoft. Acest lucru înseamnă o scalare către exterior(scale-out) a numărului de nuclee necesare pentru o aplicație. Deși există consecințe asupra design-ului unei arhitecturi software potrivite, există dovezi clare că scalarea către exterior a hardware-ului este semnificativ mai eficientă din punct de vedere al costurilor decît scalarea în sus.\newline
Microsoft a instalat data-centre Azure în 19 regiuni de pe glob, de la Melbourne până în Amsterdam sau din Sao Paulo până în Singapore. În plus, Microsoft are o înțelegere cu Via21Net, făcând astfel Azure disponibil în două regiuni din China.\newline
Azure oferă flexibilitatea de a seta foarte rapid un mediu de dezvoltare și de testare. Acestea pot fi setate folosind script-uri, oferind astfel posibilitatea de a seta un mediu de dezvoltare sau testare, de a testa, iar mai apoi de a reveni la starea inițială. Acest mecanism păstrează costurile foarte mici, iar mentenanța este aproape inexistentă.\newline
Un alt avantaj al Azure este că se pot încerca oricâte versiuni de software fără a fi nevoie de a actualiza echipamentul folosit. De exemplu, se pot verifica ramificațiile rulării unei aplicații folosind Microsoft SQL Server 2014 în locul Microsoft SQL Server 2012, se poate crea o instanță de SQL Server 2014 și se poate rula o copie a unor servicii pe noua bază de date, toate fără a fi nevoide de a actualiza hardware-ul sau de a face noi conexiuni (fizice, cu fire).\newline

\subsection{Oferta Cloud}
Cloud computing este de obice clasificat în trei categorii: Saas, Paas și IaaS. Totuși, pe măsură ce cloud crește, diferențele între aceste trei categorii încep să dispară.
\textbf{SaaS: Software as a Service}

SaaS este software ce este găzduit central și întreținut pentru consumatorul final. De obicei este bazat pe o arhitectură pe mai multe straturi – o singură versiune a aplicației este folosită pentru toți consumatorii. Poate fi scalată către exterior la multiple instanțe pentru a asigura cea mai bună performanță în toate locațiile. Software-ul SaaS este de obicei licențiat printr-o taxă lunară sau anuală.
Office 365 este un model prototip a SaaS. Subscriberii plătesc o taxă lunară sau anuală, și primesc Exchange As a Service(Outlook online și local), Storage as a Service (OneDrive), și restul suitei Microsoft Office.Subscriberii primesc de obicei ultimile versiuni de software.
Alte exemple de Saas include Microsoft One Drive, Dropbox, WordPress, sau Amazon Kindle.

\textbf{PaaS: Platform as a Service}

Cu PaaS, se pot instala aplicații în mediul de hosting oferit de către distribuitorul de servicii cloud. Programatorul dezvoltă aplicația iar distribuitorul de servicii cloud asigură abilitatea de a instala și rula aplicația. Acest mecanism uțurează viața programatorilor, și îi scutește de administrarea infrastructurii, permițîndu-le să se focuseze strict pe dezvoltare.
Azure oferă mai multe PaaS-uri, inclusiv Azure Websites sau Azure Cloud Services. Așadar dezvoltatorii au multiple variante de a instala o aplicație fără să știe nimic despre infrastructură. Aceștia nu trebuie să creeze mașini virtuale, nu trebuie să folosească Remote Desktop pentru a se loga pe acestea. 
Scalarea spre exterior a unui serviciu Azure de calcul se rezumă de obicei la creșterea numărului de instanțe , moment în care Azure instalează o nouă mașină virtuală și software-ul. Azure se ocupă chiar și de mecanismul de load-balancing(echilibrare). Pentru a instala o nouă versiune, o aplicație trebuie doar să fie republicată.

\textbf{IaaS: Infrastructure as a Service}

Un distribuitor de servicii IaaS rulează și întreține servere rulând software de virtualizare, permițând astfel creare de mașini virtuale ce rulează pe infrastructura distribuitorului de servicii cloud. În funcție de distribuitor, se pot crea mașini virtuale ce rulează Windows sau Linux și se poate instala orice pe acestea.Azure oferă posibilitatea de a configura rețele virtuale, load balancers, mecanisme de stocare, și de a folosi multiple servicii. Folosind IaaS, singurele lucruri ce nu se pot controla sunt hardware-ul și software-ul de virtualizare.
Azure Virtual Machines este oferta Azure ca și IaaS, și este o alegere populară atunci când se migrează servicii către Azure deoarece IaaS permite utilizarea modelului de intershimbare pentru migrații. Se pot configura mașini virtuale similare infrastructurii pe care rulează în prezent o aplicație și se poate migra aplicația foarte ușor.

\textbf{Servicii Azure}
Azure include multiple servicii, dar cele mai importante sunt:
\begin{itemize}
	\item Servicii de calcul: Acestea includ Microsoft Azure Cloud Services, Azure Virtual Machines, Azure Websites și Azure Mobile Services
	\item Servicii de date: Acestea includ Microsoft Azure Storage, Azure SQL Database și Redis Cache
	\item Servicii de aplicații: Acestea include servicii ce pot fi folosite pentru a construi și opera aplicații, cum ar fi Azure Active Directory, Service Bus pentru conectarea de sisteme distribuite, HDInsight pentru procesarea de date cu volum mare, Azure Scheduler și Azure Media Services
	\item Servicii de rețea: Acestea include caracteristici Azure cum ar fi Virtual Networks, Azure Content Delivery Network și Azure Traffic Manager.
\end{itemize}





