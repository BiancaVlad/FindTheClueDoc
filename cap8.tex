\chapter{Arhitectura aplicației}
Aplicația este formată din trei module importante. 
Primul modul este reprezentat de aplicația client - aplicație Android, al doilea modul este reprezentat de server - aplicatie web și aplicația care se ocupă cu preluarea și prelucrarea datelor de pe Internet și depunerea acestora în baza de date aflată în cloud - aplicație Java Spring.
\section{Aplicația client}
Aplicația client este o aplicație pentru dispozitive mobile care au sistem de operare Android. Aceasta a fost implementată cu ajutorul limbajului Java utilizând ca și mediu de dezvoltare Android Studio, astfel că am folosit o serie de instrumente interactive care se îmbină în armonie pentru a crea o interfață prietenoasă, ușor de utilizat de către deținătorii de smartphone-uri Android. 
Aplicația a fost dezvoltată având ca și nivel API minim versiune 14, acoperind astfel 87,9\% din toate dispozitivele active în Google Play Store, începând cu versiunea de Android Ice Cream Sandwich. Am ales această versiune de API datorită faptului că permite utilizarea de funcționalități noi care în versiunire anterioare nu sunt disponibile, rezultând astfel un design mai modern și mai intuitiv. Mai mult, numărul de device-uri active cu un nivel API inferior este în scădere, de asemenea și numărul de aplicații dezvoltate pentru versiunile inferioare a scăzut semnificativ în ultima perioadă.
\section{Aplicația desktop Java}
Acest modul se ocupă cu preluarea și parsarea informatiilor despre produse cu ajutorul API-ului Jsoup HTML Parser\footnote{vezi capitolul 5 - Jsoup API}.\newpage Această aplicație se ocupă preluarea produselor de pe anumite site-uri și în funcție de fiecare categorie va rezulta o listă de produse care apoi va fi filtrată și în final alegându-se doar produsele pentru care se aplică reducere, pentru a nu stoca informații pe care nu le vom folosi ulterior. Atfel după preluarea datelor de pe Internet se asigură persistența acestora prin intermediul serviciului REST. 
\section{Aplicația server}
Aplicația server este reprezentată de un serviciu REST care a fost implementat cu ajutorul templatelul Web API regăsit în ASP.NET\footnote{vezi capitolul 6 - REST}
Acesta reprezintă legătura între cele două module prezentate anterior întrucât prin intermediul serviciului se depun datele în baza de date, se preiau datele din baza de date și se modifică cu ajutorul operațiilor REST. O interfață, ce definește contractul între server și viitorii clienți, conține toate metodele la care clienții vor avea acces. Aceste metode sunt implementate în clasa serviciului. 
Persistenta datelor este realizata într-o bază de date aflată în cloud\footnote{pentru detalii despre cloud computing vezi Capitolul 7 }. Comunicarea dintre baza de date și server este efectuată folosind ORM-ul (Object Relationing Model) Entity Framework. Acest model este util deoarece întreaga bază de date este cartografiată în server. Acest lucru înseamnă că pentru fiecare tabel din baza de date, este creată o clasă iar clasele vor fi puternic conectate intre ele. De asemenea orice modificare adusă bazei de date este ușor de integrat și în server.

